
\documentclass{article}

\setlength{\oddsidemargin}{0.05 in}
\setlength{\evensidemargin}{-0.05 in}
\setlength{\topmargin}{-0.6 in}
\setlength{\textwidth}{6.5 in}
\setlength{\textheight}{9.5 in}
\setlength{\headsep}{0.25 in}
\setlength{\parskip}{0.1 in}

%
% ADD PACKAGES here:
\usepackage [usenames] {color}
\definecolor {infocolor} {rgb} {0.6,0.6,0.6}
\definecolor {sepia} {rgb} {0.75,0.30,0.15}
\everymath{\color{sepia}}
\everydisplay{\color{sepia}}
%

\usepackage{amsmath,amsfonts,amssymb,enumerate,graphicx}

%
% The following commands set up the lecnum (lecture number)
% counter and make various numbering schemes work relative
% to the lecture number.
%
\newcounter{lecnum}
\renewcommand{\thepage}{\thelecnum-\arabic{page}}
\renewcommand{\thesection}{\thelecnum.\arabic{section}}
\renewcommand{\theequation}{\thelecnum.\arabic{equation}}
\renewcommand{\thefigure}{\thelecnum.\arabic{figure}}
\renewcommand{\thetable}{\thelecnum.\arabic{table}}

\newcommand{\tran}{^{\mbox{\tiny $\top$}}}
\newcommand{\tleq}{^{\mbox{\tiny $\leqslant$}}}
\newcommand{\teq}{^{\mbox{\tiny $=$}}}

%
% The following macro is used to generate the header.
%
\newcommand{\lecture}[2]{
   \pagestyle{myheadings}
   \thispagestyle{plain}
   \newpage
   \setcounter{lecnum}{#1}
   \setcounter{page}{1}
   \noindent
   \begin{center}
       \vbox{\vspace{2mm}
         \hbox {\leftline{\Large LECTURE #1: \hfill}}
         \vspace{3mm}
         \hbox {\leftline{\Large #2 \hfill}}
         \vspace{4mm}
         \hrule
         \vspace{3mm}
         \hbox to 6.5in { {{\large Lecture Notes on combinatorics}  \hfill Mar. 2013} }
         \vspace{3mm}
        }
   \end{center}
   \markboth{LECTURE #1: #2}{LECTURE #1: #2}
   \pagenumbering{arabic}
   \vspace*{4mm}
}
%
% Convention for citations is authors' initials followed by the year.
% For example, to cite a paper by Leighton and Maggs you would type
% \cite{LM89}, and to cite a paper by Strassen you would type \cite{S69}.
% (To avoid bibliography problems, for now we redefine the \cite command.)
% Also commands that create a suitable format for the reference list.
\renewcommand{\cite}[1]{[#1]}
\def\beginrefs{\begin{list}%
        {[\arabic{equation}]}{\usecounter{equation}
         \setlength{\leftmargin}{2.0truecm}\setlength{\labelsep}{0.4truecm}%
         \setlength{\labelwidth}{1.6truecm}}}
\def\endrefs{\end{list}}
\def\bibentry#1{\item[\hbox{[#1]}]}

%Use this command for a figure; it puts a figure in wherever you want it.
%usage: \fig{NUMBER}{SPACE-IN-INCHES}{CAPTION}
\newcommand{\fig}[3]{
			\vspace{#2}
			\begin{center}
			Figure \thelecnum.#1:~#3
			\end{center}
	}
% Use these for theorems, lemmas, proofs, etc.
\newtheorem{theorem}{Theorem}[lecnum]
\newtheorem{lemma}[theorem]{Lemma}
\newtheorem{proposition}[theorem]{Proposition}
\newtheorem{claim}[theorem]{Claim}
\newtheorem{corollary}[theorem]{Corollary}
\newtheorem{definition}[theorem]{Definition}
\newenvironment{proof}{{\it Proof.}}{ \hfill $\square$}

% **** IF YOU WANT TO DEFINE ADDITIONAL MACROS FOR YOURSELF, PUT THEM HERE:

\def\R{{\mathbb R}}

\begin{document}
%FILL IN THE RIGHT INFO.
%\lecture{**LECTURE-NUMBER**}{**DATE**}{**LECTURER**}{**SCRIBE**}
\lecture{11}{Equivalence of some theorems in combinatorics}
%\footnotetext{These notes are partially based on those of Nigel Mansell.}

% **** YOUR NOTES GO HERE:

% Some general latex examples and examples making use of the
% macros follow.
%**** IN GENERAL, BE BRIEF. LONG SCRIBE NOTES, NO MATTER HOW WELL WRITTEN,
%**** ARE NEVER READ BY ANYBODY.

\textbf{Crossreference }

Dilworth

K\"{o}nig Theorem




\begin{theorem}[Dilworth's Theorem]
The minimum number of chains into which the elements of a partially ordered set $P=(X,\prec)$ can be partitioned is equal to the maximum number of elements in an anti-chain of P.
\end{theorem}

partition = disjoint covering
induction proof

Dual of Dilworth’s Theorem


Dilworth Theorem via K\"{o}nig Theorem

\begin{proof}
To prove Dilworth's theorem for a partially ordered set $P$ with $n$ elements, using K\"{o}nig's theorem, define a bipartite graph $G[U,V]$ where $U=V=X$ and where $(u,v)$ is an edge in $G$ when $u\prec v$ in $P$. 

First, we show that any matching $M$ of $G[U,V]$ yields a chain partition of $P$. Let $Z$ be the family of chains formed by including $u$ and $v$ in the same chain whenever there is an edge $(u,v)$ in $M$. In particular, when $M=\emptyset$, each chain of $Z$ is a singleton element in $X$. Obviously, $Z$ is proper chain partition of $P$ and $\lvert Z\rvert=n-\lvert M\rvert$.

Then we prove that any vertex cover $C$ gives rise to a anti-chain with $n-\lvert C \rvert$. Let $A$ be the set of elements of $X$ that are not in $C$. Clearly, $\lvert A \rvert=n-\lvert C \rvert$. Moreover, $A$ is an anti-chain, since an comparable pair in $A$ will lead to an edge of $G[U,V]$ uncovered by $C$.

Since $\lvert M \rvert \leq \lvert C \rvert$, we have $n-\lvert C \rvert \leq n- \lvert M \rvert$. By K\"{o}nig's theorem, there are a maximum matching $M^*$ and minimum vertex cover $C^*$ in $G[U,V]$ such that $\lvert M^* \rvert = \lvert C^*\rvert$. So $n-\lvert C^* \rvert = n- \lvert M^* \rvert$.
\end{proof}


\section*{References}
\beginrefs
\bibentry{CW87}{\sc D.~Coppersmith} and {\sc S.~Winograd},
``Matrix multiplication via arithmetic progressions,''
{\it Proceedings of the 19th ACM Symposium on Theory of Computing},
1987, pp.~1--6.
\endrefs

% **** THIS ENDS THE EXAMPLES. DON'T DELETE THE FOLLOWING LINE:

\end{document}







